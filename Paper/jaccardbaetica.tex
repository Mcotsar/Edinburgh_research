\documentclass[review]{elsarticle}
\usepackage{hyperref,lineno}
\usepackage{wrapfig}
\usepackage{lscape}
\usepackage{rotating}
\usepackage{xcolor}
\usepackage[utf8]{inputenc}
\modulolinenumbers[5]


\newcommand{\memo}[2]{\textcolor{#1}{#2}}
\newcommand{\maria}[1]{\memo{red}{MC: #1\\}}
\newcommand{\xavi}[1]{\memo{magenta}{XRC: #1\\}}

\journal{Journal of Archaeological Method and Theory}


\bibliographystyle{model2-names.bst}\biboptions{authoryear}


\begin{document}

\begin{frontmatter}

%\title{The markings of the trade: exploring the patterns of olive oil production in Roman Baetica}
\title{The tracing of trade: exploring the patterns of olive oil production and distribution from Roman Baetica}

\author[ceipacadress]{María Coto-Sarmiento\corref{mycorrespondingauthor}}
\cortext[mycorrespondingauthor]{Corresponding author}
\ead{mcotsar@gmail.com}


\author[edadress,ubadress]{Xavier Rubio-Campillo}

\address[ceipacadress]{Department of Prehistory and Archaeology, Montalegre, 6-8, 08001, Universitat de Barcelona, Barcelona, Spain}
\address[edadress]{School of History, Classic \& Archaeology, Room OOM.33, William Robertson Wing, Old Medical School, Teviot Place, University of Edinburgh, UK}
\address[ubadress]{DIDPATRI, Universitat de Barcelona, Passeig de la Vall d'Hebrón, 171, Barcelona, Spain}

\begin{keyword}
Roman Empire; amphora production; Dressel 20; dissimilarity index; Roman provinces
\end{keyword}


\begin{abstract}

The aim of this study is to explore economics dynamics in the production and distribution of olive oil trade in \textit{Baetica} province (currently Andalusia). \textit{Baetica} became an important production and distribution centre during the Roman Empire. However, it remains under debate about how this province was organised and whether it could be possible to identify patterns in the olive oil market. 

Our case of study has been focused on amphoric production from 1st to 3rd centuries AD. In particular, we want to detect patterns of olive oil production that link amphora workshops and amphoric stamps. Amphoric stamps are used to identify the presence of different groups that might share similar stamps. 

To achieve this goal, we analyse a set of stamps from two types of centres (production and consumption): 1) production centres by analysing different workshops in \textit{Baetica} province and 2) two Roman provinces such as \textit{Germania} (Inferior and Superior) and \textit{Britannia} as consumption centres. They will be used to detect a connection between the distribution of amphoric stamps and the economic structure in both centres. Here, we use methods borrowed from Ecology that allow us to identify if amphora workshops share similar amphoric stamps depending on the spatial distance. The analysis explores how the quantitative approach provides a useful tool for the interpretation of the economic processes. Finally, results pretend to highlight the organisation of Baetican olive oil production in the Roman Empire linked to the differences observed in the archaeological evidence.


%Our case of study has been focused on the production processes located in \textit{Baetica} province (currently Andalusia) from 1st to 3rd AD. In particular, we want to detect patterns of olive oil production that link amphora workshops and amphoric stamps. \textit{Baetica} became an important production and distribution centre during the Roman Empire. However, it remains under debate about how this province was organised and whether it could be possible to identify patterns in the olive oil market. Amphoric stamps are used to identify the presence of different groups that might share similar stamps. To achieve this goal, we analyse a set of stamps from two centres: 1) production centres by analysing different workshops in \textit{Baetica} province and 2) two Roman provinces such as \textit{Germania} and \textit{Britannia} as consumption centres. They will be used to detect a connection between the distribution of amphoric stamps and the economic structure in both centres. Here, we use methods borrowed from Ecology that allow us to identify if amphora workshops share similar amphoric stamps depending on the spatial distance. The analysis explores how the quantitative approach provides a useful tool for the interpretation of the economic processes. Finally, results pretend to highlight the organisation of Baetican olive oil production in the Roman Empire linked to the differences observed in the archaeological evidence.

\end{abstract}


\end{frontmatter}


\section{Introduction}

The intensification of long-range trade was one of the most important traits of the economy developed during the Roman Empire. The development of an extensive road network increased the connectivity between inland communities, but most shipping continued being based on maritime routes particularly in the Mediterranean basin~\citep{temin_market_2001,bevan_mediterranean_2014}. The intensity of this trade between distant regions can be observed both in archaeological and written evidence~\citep{rodriguez_baetican_1998}.

%Material culture is one of the most frequent indicators of trade in the archaeological record. In archaeology, they allow us to highlight a part of the mechanism of production and distribution of goods along the Mediterranean \citep{bevan_mediterranean_2014}. Particularly, the spread of these factors had an important impact during the Roman Age, when the progressive exploitation of communication networks allowed a major interaction between communities \citep{orengo_seeds_2016} 
%spread of goods and ideas had a enormous impact during the Roman Empire. The progressive exploitation of communication networks allowed a major interaction between communities
%This can be seen by the fact that material culture is found in different regions along the Mediterranean as a frequent indicator of trade exchange in the archaeological record. 

The Empire developed a series of structures to support and organise this long-range trade and specialised entire regions to massively produce specific goods. An example of this process was the province of \textit{Baetica} (currently Andalusia, southern Spain) which became an important olive oil production centre during the Roman Empire. Olive oil was an essential good for Romans because it was used in almost every aspect of their daily life such as cooking, hygiene or lighting~\citep{mattingly_d.j._oil_1988}. This high demand required a huge increase on the production which was distributed through amphorae shipped via maritime routes to all the provinces and particularly to Italy and the hundreds of garrisons that the Roman army deployed along its borders~\citep{blazquez_exportacion_1980}. 

The structure and processes associated with this massive olive oil production have been extensively discussed over the last decades\citep{rodriguez_economioleicola_1977, Chic_hispania_1997,millet_anforas_1998}. 
However, the lack of written records has not made possible to detect any indication of patterns in the olive oil market.

Our knowledge of this economic activity has benefited from new findings and data sources, but despite all these advances several questions remain open: How was amphora production organised during the Roman Empire? Can we detect economic dynamics between provinces by analysing the distribution of amphorae?

Advances in the research of the Roman studies have currently led an environment with various interpretations on commercial dynamics \citep{duncan1982economy,
temin_economy_2006,
quantifyingwilson2009}.
%\xavi{more diverse que qué?} 
The application of different quantitative approaches has allowed us to improve our interpretation of the growing amount of archaeological evidence and reveal the dynamics of Roman production
\citep{brughmans_roman_2016,
orengo_seeds_2016,bayesian_2018,
coto-sarmiento_identifying_2018,
rubio-campillo_ecology_2018}.

This work focuses on the debate about the dynamics of the distribution of olive oil market in Roman provinces. Amphoric production in \textit{Baetica} province will help us identify links between production centres and consumption centres in order to detect an economic pattern in the commerce by analysing stamps amphorae finding in the provinces. 

The finding of amphoric stamps in different provinces has allowed us to know the importance of the distribution of olive oil within the Empire. Despite having archaeological evidence, no consensus has been reached: How was the economical relation between production and consumption centres? Did they follow an economical pattern to distribute olive oil in those provinces?
  
This paper aims to study the olive oil market connection between provinces by calculating the similarity of stamps. Specifically, our work pretends to detect regional processes focused on a commercial product from a specific province \citep{isaksen_network_2006}. We want to understand the pattern of olive oil production linked to amphora workshops and amphoric stamps used to mark them. We focus here on exploring the economic relation between stamps and amphora production and distribution centres. 

Two case studies have been studied in order to analyse the relation between production centres (\textit{Baetica}) and consumption centres (\textit{Britannia} and \textit{Germania}) (see Figure~\ref{general}).

\begin{figure}[htp]
	\centering
\includegraphics[width=\linewidth]{figs/general_map}
\caption{Overview of the sites analysed in this paper. The colour defines the three different regions under study (\textit{Baetica} as producer; \textit{Germania} and \textit{Britannia} as consumers) while the size of each dot highlights the number of Dressel 20 amphoric stamps found on each of these sites}

\label{general}
\end{figure} 
        
In the case of \textit{Baetica} province, we want to identify the role of the stamps in the organisation of the workshop; in Roman provinces, our aim is detecting groups of stamps concentrated in an area or if some groups have an important role for the exportation of olive oil in those provinces. This economic connection could be identified by different aspects: a) correlation between spatial distance and the sites based on the idea that closer workshops concentrate similar amphoric stamps in a specific area than the distant workshops and b) groups of similar stamps were concentrated in a specific province. 

In particular, we study the distribution of amphoric stamps to identify a correlation between geographical distance and similarity. Based on this assumption, we proposed three hypotheses: a) we can identify a correlation between spatial distance and the distribution of stamps, b) stamps located in close workshops share similar traits and c) low mobility of amphoric stamps to other regions: stamps always stay in the same region. 


This study proposes a robust baseline to explore the distribution of Baetican olive oil production by computing the spatial correlation between stamps. To do this, a population approach has been used to analyse the dispersion of stamps between amphora workshops \citep{rubio-campillo_ecology_2018}.  A way to analyse is to use a quantitative framework to measure the similarity between amphoric stamps. Here, we use an ecological approach based on three steps: a) to detect similarities between stamp codes, b) to explore a potentially spatial correlation and c) to establish a correlation between similarity of stamps and spatial distance. Stamps will be used to identify economic patterns by analysing their similarity. If workshops and provinces share stamps with similar traits, then we can identify connections. By contrast, if we do not detect similar stamps between workshops, then workshops worked independently. 

The paper addresses these questions as follows: the next section introduces the historical context. Section two displays the dataset and the methods used for the analysis. Section three presents the results and the last section shows the discussion and the main conclusions of this work. 


\section{The Amphoric production of Baetica}

The landscape of the \textit{Baetican} province was one of the best regions to face the increased demand for olive oil across the Roman provinces. For this reason, the area saw an increase in productivity as a massive infrastructure of olive oil production was gradually deployed. The production and distribution of olive oil increased from the first to the third centuries AD~\citep{remesal_concierto}. 


Hundreds of presses and amphora-making workshops were built near large extensions covered by olive trees. The workshops where amphorae were made and filled with olive oil were places along the rivers Guadalquivir and Genil riverbanks (see Fig.\ref{workshop}).

This strategic location allowed the transport of olive oil through riverine shipping towards the maritime routes that connected \textit{Baetica} with Mediterranean and Atlantic trade routes towards the rest of the Empire~\citep{garcia_vargas_enrique_formal_2010}.

\begin{figure}[htp]
	\centering
\includegraphics[width=\linewidth]{figs/baetica}
\caption{Distribution of stamps found in Dressel 20 workshops along the Guadalquivir riverbank. The size of each point depends on the number of stamps found on each site}

\label{workshop}
\end{figure} 


The chronology of the workshops is wide spanning from the first to the third centuries AD~\citep{millet_anforas_1998,rodriguez_baetican_1998,chic2005comercio}. 
This fact is shown by the archaeological evidence that displays a highly specialised production with a long activity \citep{remesal_anforas_2004}. Thus, the specialisation of this production made amphorae with apparently few changes in their forms, reaching a high degree of standardisation. 

The importance of this olive oil trade is revealed by the massive amount of Dressel 20 type amphorae found across the Roman provinces \citep{dressel_ricerche_1878,
millet_anforas_1998}. This amphora has been commonly associated with the transportation of Baetican olive oil for supplying military camps and civil settlements \citep{berni_millet_epigrafianforica_2008}.

They have a recognisable globular form with a short neck and oval-shaped handles. Its capacity allowed to hold 70-75 litres of olive oil to be shipped and distributed throughout Empire \citep{berni_dressel_2016}.


Dressel 20 amphorae display a lot of evidence, but some of it is still hard to interpret. A large percentage of Dressel 20 were marked with stamps, while they could also be inked with \textit{tituli picti} or incised with \textit{graffiti}. The interpretation for most of these inscriptions is still open due to the fragmentation of the material and the small sample size of well-preserved elements~\citep{aguilera_evolucion_2007,rovira_guardiola_grafitos_2007}. 

%Some stamps show a more specific chronology while the majority of them display a large activity of production that it can be difficult for specifying an accurate chronology in some cases. This could be due to two reasons; firstly, most of the workshops were partially excavated and focused on archaeological surveys in order to collect the maximum stamps as possible; secondly, Dressel 20 was produced during almost three centuries with apparently few changes \citep{berni_piero_chapter_2017}.
 

\section{A potential indicator of Roman economy: Dressel 20 stamps}

Dressel 20 was the amphora most stamped during the Roman Empire ~\citep[18]{millet_anforas_1998}. The large sample size of recovered Dressel 20 stamps is distributed across all the western provinces of the Roman Empire. For this reason, there are several publications of this type of evidence discussing its origin, its meaning, its long period of use and its spatial distribution~\citep{dressel_ricerche_1878,
rodriguez_economioleicola_1977,
chicepi1985,millet_anforas_1998, remesal_sellar_2016} (Fig.\ref{amphora}).


\begin{figure}[htp]
	\centering
\includegraphics[scale=0.5]{figs/dressel20}
\caption{Dressel 20 were mostly marked with stamps of three letters called \textit{tria nomina}}
\label{amphora}
\end{figure} 

Most stamps display a large activity of production that can be difficult to date using an accurate chronology. They have been frequently dated with the consular dating by studying \textit{tituli picti} found in \textit{Monte Testaccio} \citep{Testaccio1, berni_millet_epigrafianforica_2008}.
Despite \textit{tituli picti} contain important information about chronology, they have been still found in few quantities mostly concentrated in an only settlement such as \textit{Monte Testaccio}. In addition, the chronology of stamps could be also biased taking into account that amphorae were deposited after being produced and consumed.

The use of amphoric stamps can be a good proxy to explore the system of olive oil production and trade developed in \textit{Baetica} province. These stamps are typically found on handles but they were also imprinted on rims and the amphora body~\citep{millet_anforas_1998}. The content information of the stamps is shown in different forms and letter content and it seems that there was not a unique criterion.

Stamps often displayed a code of three letters and they are known as \textit{Tria Nomina} \citep{berni_millet_amphora_1996}. The three letters correspond to initial forms of a name but they can also appear complete. 

Scholars agree that stamps are some type of identification mark, but there is no consensus on the meaning of these marks~\citep{rodriguez_baetican_1998}. 

%Additionally, a large percentage of Dressel 20 containers were not stamped thus complicating the interpretation 

The stamp codes are interpreted based on three main ideas: content (olive oil), context (amphora workshop) and subject (individuals involved in the production). On the one hand, it seems that stamps could have been identified as the landowner of the olive groves \citep{rodriguez_economioleicola_1977}. On the other hand, they could also belong to the owner of the workshops where the amphorae were made or even to a group of amphora workers~\citep{berni_millet_epigrafianforica_2008}. 
%In any case, the use of these stamps is a good proxy to explore the system of olive oil production and trade developed in Baetica.

Nevertheless, some challenges remain under discussion such as how this production was organised and whether it is possible to distinguish production patterns in the olive oil trade. Our questions will be focused on the distribution of amphoric stamps. Did they follow a distribution pattern? Did stamps share the same workshop? 


%\subsubsection{Jaccard distance}
%The dataset was analysed using a statistic method as Jaccard distance. This method allows to measure the dissimilarity by calculating the presence of sets (CITAR). In our case, Jaccard distance was used to compute the mutual presence of traits in the amphora stamps but it does not consider the number of absences. A comparison was done with the distance of the workshops to identify whether there was an association between stamps and spatial distance amongst workshops. 

\section{Consumption centres: Britannia and Germania}

The creation of new provinces required the Roman Empire the arrival of resources through Mediterranean and Atlantic routes.

This led to a gradual change in the economic and social structures the trade networks that supplied the Roman army became more complex. Augustus's administration created the figure of the \textit{praefectura annonae} to organize wheat supply. The role of the \textit{praefectura annonae} was mainly focused on providing through \textit{frumentationes} a fixed monthly amount of wheat to each Roman citizen~\citep{remesal_annona_1986,remesal_concierto}

However, some hypotheses argue elsewhere that this same \textit{praefectura annonae} could also have organised the supply of additional goods such as olive oil to the Roman legions  \citep{remesal_annona_1986,
remesal_annona_1990}


%The importance of this olive oil trade is revealed by the massive amount of Dressel 20 amphorae found across the Roman provinces. This amphora has been commonly associated with the transportation of Baetican olive oil for supplying military camps and civil settlements \citep{berni_millet_epigrafianforica_2008}.\xavi{me parece repetitivo, porque unos párrafos antes ya hablabas de lo mismo. Quizás unificar texto aquí o allá?} LO HE PUESTO MÁS ARRIBA!
 

Olive oil supply seems to be particularly intense in militarised provinces that functioned as the borders of the Empire such as \textit{Britannia} and \textit{Germania}. The high percentage of Dressel 20 amphorae found in sites nearby military garrisons suggests the existence of diverse supply dynamics as there were trade processes flowing from military sites to civil settlements~\citep{remesal_annona_1986, carreras_britannia_1998}.

An explanation for the interest to export olive oil for the military garrison in the Roman provinces is still unclear. The origin of this exportation may be related to two aspects: a) cultural consumption whose olive oil is used by cultural reasons such as group identification or by habit and b) economical and political reasons to create a redistributive system mechanism to supply military camps \citep[69-70]{carreras_britannia_1998}. 

The Roman conquest in \textit{Britannia} and \textit{Germania} was a new opening to foreign trade in order to supply the military campaign. High demand for olive oil in both provinces stimulated an important trade network involving different agents and the Empire itself. \textit{Baetica} became a strategic province for the production and distribution in both provinces since it can be detected by the high presence of amphoric stamps found in civil and military areas during this period. 



\subsection{Britannia}

The evidence for olive oil consumption in \textit{Britannia} is scarce before the Roman conquest \citep{funari_corpus_1996,carreras_abastecimiento_2003}. The local population did not consume this product and it was not produced on the region as the landscape and climate of the British Isles was not suitable for olive oil trees\citep[161]{monfort_britanniaen_1998}. Based on the archaeological records, this absence of olive oil changed with the arrival of a large amount of Dressel 20 amphorae to \textit{Britannia} from the first century A.D \citep{peacock_amphorae_1991,
carreras_britannia_1998}. At this moment, we detect an increase of the olive oil exportation concurring with the displacement of legions during the military campaigns \citep[161]{monfort_britanniaen_1998}.

The increase of the exportation of Dressel 20 amphorae created an important commercial network for exchanges. The demand for olive oil after the Roman conquest required of a new trade route due to the lack of local olive oil production; \textit{Baetica} was chosen as its main supply.

%At this moment, we detect an increase of the olive oil exportation concurring with the displacement of legions during the military campaigns \citep[161]{monfort_britanniaen_1998}\xavi{repetitivo? fusionar párrafos}.

This fact will have a particular intensity within sites close to the Hadrian Wall's garrisons. Olive oil production in \textit{Baetica} would cross the Atlantic until they reached the province and redistribute throughout the area from a series of strategic locations~\citep{carreras_atlantic_2012}. 

%The increase of the exportation of Dressel 20 amphorae created an important commercial network for exchanges. Thus, the network was mainly focused on the support of soldiers during military campaigns\xavi{estas 2 frases repiten cosas ya explicadas}. 

The presence of Dressel 20 stamps in military camps in \textit{Britannia} has been widely studied in Roman archaeology \citep{williams_importation_1983,
funari_corpus_1996,
carreras_britannia_1998,carreras_abastecimiento_2003}.

This intense consumption of olive oil also indicates a possible provincial structure designed to organise the supply of olive oil to military camps. 

There are no written records explaining how this redistribution of essential goods was organised in \textit{Britannia}, but the archaeological evidence suggests that cities may have been the central nodes of this local trade network~\citep[45]{funari_economic_2005}.

The consumption of olive oil would experience a progressive slowdown from the third century A.D. onwards. This date matches a change in market strategy in the Empire. This gradual decrease on \textit{Baetican} olive oil exports can be observed as a decrease in the amount of Dressel 20 found in excavated sites as this typology is replaced by other types of amphorae from other places \citep{rodriguez1991aceite,millet_anforas_1998}.


\subsection{Germania}

Roman studies exploring the presence of Dressel 20 amphorae in the provinces Inferior and Superior during Roman Empire have not had the same impact as the rest of the provinces \citep[293]{remesal_baetica_2002}.

The massive importation of Dressel 20 in both \textit{Germania} provinces could be explained due to the archaeological sources found mostly in German \textit{limes}. 
By contrast, it is still unknown if trade agents participated in the distribution of Baetican olive oil in \textit{Germania} or the supply was exclusively focused on the Roman army garrisons\citep[156]{remesal_germn_2010}.

%This could be explained due to the lack of archaeological sources \citep{horacio2010llegada}

\textit{Germania} presents a similar introduction pattern of olive oil than the one discussed for \textit{Britannia}. There is no archaeological evidence for olive oil consumption before the arrival of the Roman army to the region while a majority of recovered Dressel 20 amphorae are located in the military sites that formed the German \textit{limes} \citep{remesal_germaniaengl_2002}.

The presence of the Roman army encouraged the exchange in the province as shown by the arrival of this product both civil settlement and military sites with a major concentration at the German \textit{limes}.

It seems that some Baetican centres could have been assigned to a specific province for the olive oil supply to military garrisons \citep[125]{remesal_concierto}. However, this hypothesis is hard to assess given the current lack of written evidence related to the meaning of the amphoric stamps. 

Previous studies suggested that the supply to the German limes was mainly based on riverine transport, but recent works suggest that the maritime route through the Atlantic Ocean could have been more important than expected (both for \textit{Germania} and \textit{Britannia})~\citep{remesal_germn_2010,rubio-campillo_ecology_2018}.


\section{Material and Methods}

The goal of this study is exploring the effect of the production patterns between different centres. We are especially interested in identifying links between production and consumption centres by using amphoric stamps. 

To study the amphoric stamps, we used the data collected in the CEIPAC database of amphoric epigraphy from different places~\citep{remesal_centro_2015} (see the database here \url{http://romanopendata.eu}). The CEIPAC dataset contains over 50.000 epigraphy records found on different types of amphorae, mostly from \textit{Monte Testaccio}. 

%\xavi{podrías mover esta parte a los 2 párrafos anteriores y explicar la bbdd brevemente antes de empezar a hablar de la Bética (porque es info que también aplica a los otros 2 casos de estudio no?}.


%This study proposes a robust baseline to explore the distribution of Baetican olive oil production by computing the spatial correlation between stamps. A way to analyse is to use a quantitative framework to measure the similarity between amphoric stamps. Here we use an ecological approach based on three steps: a) to detect similarities between stamp codes, b) to explore a potential spatial correlation and c) to establish a correlation between similarity of stamps and spatial distance. 
%\xavi{estos 2 párrafos explican objetivos, que deberían ir en la intro y no aquí, donde deberías empezar a hablar directamente de la BBDD}


\subsection{Production centres: Baetica province}


The database allowed us to retrieve 3798 stamps found in Dressel 20 amphorae recovered from workshops within \textit{Baetica} province. Each of these stamps ha detailed records on the site where it was found, the inscribed code as well as its spatial coordinates. The stamps with incomplete information were discarded, thus finishing with a sample of 987 stamps from 81 sites and displaying 130 different code.

%Approximately 70 \% of stamps cannot be tested due to the fragmentation of the dataset. Consequently, any stamps with incomplete information were discarded and not integrated into our dataset. 

Our dataset also contained a new categorical variable defining the \textit{conventus} of each site. The \textit{conventus} were administrative centres for territorial organisation under the Roman Empire \citep[58]{ozcariz_gil_administracion_2013}.
The creation of the \textit{conventus} in some Roman provinces was a form of control with the purpose of organising the administration in the provinces \citep{albertini_les_1923}.


The production area for Dressel 20 amphorae extended across three different \textit{conventus}: \textit{Hispalensis} (currently Seville, hereafter \textit{Hispalis}), \textit{Cordubensis} (currently C\'ordoba, hereafter \textit{Corduba}) and \textit{Astigi} (currently Écija, Sevilla, hereafter \textit{Astigi}) \citep{rodriguez_economioleicola_1977,chicdatos2001,berni_millet_epigrafianforica_2008} . 

%3798 = base de datos sin limpiar
%3791 = base de datos limpiada con cleanstamp.py
%3787 = no sé a qué corresponde pero es archivo baetica.csv

It is important to mention that some workshops exhibited the same geographical coordinates while they were catalogued differently as workshops. For instance, two nearby amphora workshops such as Tesorillo de Doña Mencia and Doña Mencia were catalogued as a different name in spite of having the same geographical coordinates. Our work considered this particularity in the archaeological dataset but some biases can occur in the analysis. 

The first step was to compute the frequency distribution of stamp codes and analyse it using Exploratory Data Analysis (EDA). This would allow us to study the distribution of the stamps across centres as can be seen in Figure~\ref{stamps}. Most workshops concentrated few stamps while few workshops contained more than 100 stamps. For instance, one workshop (La Catria) concentrated a large percentage of the amphoric stamps with a total number of 228 stamps.

The type of distribution is also frequent in amphora production where we observed a major concentration of the number of stamps in few workshops~\citep{bayesian_2018,coto-sarmiento_identifying_2018}.


%\xavi{no lo veo; recuerda que para el rollo bayesiano tenías una línea recta al hacedr el log de los 2 ejes, no solo 1, y este plot no está construído así no?}.
%\maria{son dos ejes, no? ¿aquí no se ve línea recta? Lo mismo si lo hago de puntitos...}

\begin{figure}[htp]
	\centering
\includegraphics[width=\linewidth]{figs/frequencystamp.pdf}
\caption{Histogram on a log scale with base 10. X axis is represented by the number of stamps and Y axis is the frequency of workshop. The distribution is widely diverse with most workshops having only one stamp}
\label{stamps}
\end{figure} 


The distribution of code stamps for each \textit{conventus} can be seen in Figure~\ref{frequency}. The majority of stamps are concentrated in \textit{Hispalis} (574 code stamps) while \textit{Corduba} and \textit{Astigi} have roughly half this sample size (267 and 146 code stamps). The workshops of the three \textit{conventus} show a similar pattern on the stamp frequency distribution with the exceptions of 2 large workshops in \textit{Hispalis} (La Catria and Arva). On these two workshops a comparatively large amount of code stamps were found (29 different code stamps on each of them).

According to previous studies, those workshops became the most important centres of amphora production in the region \citep{rodriguez_economioleicola_1977,
arva_1997}.

It is worth mentioning that a majority of these stamps were collected during field surveys with no excavation, so this difference in sample size could be biased due to differences in intensity across the different sites~\citep{arva_1997}.
 
\begin{figure}[htp]
	\centering
\includegraphics[width=\linewidth]{figs/frequency}
\caption{Distribution of the number of different code stamps (X axis) for each \textit{conventus} (Y axis). Each dot corresponds to a workshop sorted by different areas. Colours are represented by areas divided into \textit{Hispalis} (red), \textit{Astigi} (green) and \textit{Corduba} (blue)}
\label{frequency}
\end{figure} 


\subsection{Consumption centres: Britannia and Germania}

The analysis of stamps in consumption areas also used the CEIPAC database and filtered in the same way than the previous dataset. However, in this case, we selected the centres with more or equal than five stamps in order to maximise the analysis with more stamps in the dataset.   

The output was a dataset of 4271 stamps found in sites belonging to \textit{Britannia} (total: 2219 stamps, analysed: 1765 stamps ) and \textit{Germania} (total: 2052 stamps, analysed: 1621 stamps). Some stamps from \textit{Britannia} and \textit{Germania} were discarded for the analysis for incomplete information.  

Both \textit{Germania} and \textit{Britannia} were analysed as borders and not as Roman provinces. This means that we included in the sample some centres that are actually located beyond the administrative boundaries but were considered part of the same border. Specifically, for the German \textit{limes} we included sites from the two provinces (Superior and Inferior) while the analysis of \textit{Britannia} extended to \textit{Caledonia}.
 
%We analysed a dataset of 2219 stamps from different centres in \textit{Britannia}. 
%CITAR (Callender, 1965; Carreras Monfort y Funari,1998; Ayllón-Martı́ et al., 2018). 
In \textit{Britannia}, we studied a total sample of 1765 stamps from 46 sites displaying 968 unique codes. The centres located in \textit{Britannia} can be seen in Figure~\ref{britannia}.
 
\begin{figure}[htp]
	\centering
\includegraphics[width=\linewidth]{figs/britannia}
\caption{Sites in \textit{Britannia} where Dressel 20 amphoric stamps have been found. Each dot is a site while its size shows the sample size of stamps found on the site. A majority of stamps have been found in garrisons related to Hadrian's and Antonine walls}
\label{britannia}
\end{figure} 


%We analysed a dataset of 2052 stamps placed in \textit{Germania}. All the data was also compiled by CEIPAC database. As previously before, the centres were selected with more or equal than five stamps.

In \textit{Germania}, we collected a total of 1621 stamps from 46 sites displaying 850 different stamp codes (see Figure~\ref{germania}). 


\begin{figure}[htp]
	\centering
\includegraphics[width=\linewidth]{figs/germania}
\caption{Sites in Germania where Dressel 20 amphoric stamps have been found. Each dot is a site while its size shows the sample size of stamps found on the site. Most sites are located along the German \textit{limes}}
\label{germania}
\end{figure}


\subsection{Measuring the dissimilarity}


%quizás habría que hablar también del filtrado de códigos que se hizo en python porque antes había 3783 stamps pero se filtró por el número de letras...
%\xavi{lo del filtrado lo puedes poner de supplementary information con referencia al antiquity}

The approach proposed here aimed at exploring links between production and consumption centres through the identification of common amphorae codes found in different sites. For this reason we measured the similarity between amphora workshops and/or consumption sites by quantifying a pairwise distance or dissimilarity index between places (i.e. to what extent the stamp codes found on these sites were different). The chosen dissimilarity measure was the Morisita-Horn index \citep{morisita_measuring_1959, horn_measurement_1966}. This method was applied to measure the dissimilarity between different samples of sets. Generally, it describes the dissimilarity between the system of two communities based on the idea of inverse correlation between diversity and species \citep{magurran_why_1988}. In our case, stamps from different sites will be used as a proxy to measure their dissimilarity by analysing their code. 

The formula can be described as follows \citep{magurran_measuring_2013}:

\begin{equation}
D(MH) = 1- \frac{2 \sum(a_{i} \cdot b_{i})}{(d_{a} + d_{b}) \cdot (N_{a} \cdot N_{b})}
\end{equation} \\

$d_{a}$ and $d_{b}$ are given by the following equation:

\begin{equation}
d_{a} = \frac{\sum a_{i}^{2}}{N_{a}^{2}} 
\end{equation} \\

where $N_{a}$ is the total number of stamps in workshop A; $N_{b}$ is the total number of stamps in workshop B; $a_{i}$ is the number of different stamps for workshop A and $b_{i}$ is the number of different stamps for workshop B.

Considering our dataset as a non-uniform sample, this method provides a useful tool to handle large samples with different sizes and diversity \citep{wolda_similarity_1981}. Morisita-Horn index gives a value between 0 (sites have exact stamp codes and frequency distribution) and 1 (complete difference between stamp codes). To apply Morisita-Horn index, we first calculated the number of times that each stamp code appeared in an amphora workshop. 
This method allowed us to bear in mind the number of frequency of stamps for each workshop. 

If two workshops had similar stamp code distribution then the distance index would be close to 0 whereas sites with completely unrelated stamp codes would give a distance close to 1.

\subsection{Hierarchical clustering}

The Morisita-Horn index was used to generate a dissimilarity matrix containing the distance value between each pairwise site based on their code stamps. Hierarchical clustering was applied to this matrix in order to group sites with similar stamps codes distributions. The algorithm was selected to cluster similar groups in order to analyse the relationship between groups of sites and the distribution of similar stamp codes. The results were visualised using a dendrogram to detect groups of sites sharing similar stamp codes.  

\section{Results}

\subsection{Production centres: Baetica Province}

%The correlation coefficients range from a minimum to a maximum.

The hierarchical clustering of the Morisita-Horn pairwise distance matrix can be seen as a dendrogram in Figure~\ref{dendro}. This visualisation suggests that each amphora workshop often used unique stamps in their production system that have not been found on any other site. In fact, the similarity values are always low and a majority of stamp codes found on multiple sites are only present in two or three workshops.

Nearby workshops display a higher similarity than distant workshops and the varying degrees of similarity between sites which could be correlated to their spatial distance; additionally, the workshops with higher similarity values belong to the same \textit{conventus} area, such as Picachos, Cerro de los Pesebres and El Castillejo.

\begin{sidewaysfigure}[htp]
	\centering
\includegraphics[angle=180, width=\linewidth]{figs/dendro}
\caption{Dendrogram displaying the results of hierarchical clustering based on Morisita-Horn metrics for Dressel 20 workshops located in the \textit{Baetica} area. Site names are colour-coded by \textit{conventus}: \textit{Hispalis} (red), \textit{Astigi} (green) and \textit{Corduba} (blue)}
\label{dendro}
\end{sidewaysfigure} 


\subsection{Consumption centres: Britannia and Germania}

Results for the consumption regions display similar patterns than the ones found for the production area of \textit{Baetica}. The most important difference is that the similarity values are lower as can be seen in Figure~\ref{britmap}.


\begin{sidewaysfigure}[htp]
	\centering
\includegraphics[angle=180,width=\linewidth]{figs/dendrobrit5.pdf}
\caption{Dendrogram displaying the results of hierarchical clustering based on Morisita-Horn metrics for sites in \textit{Britannia}. Site names are colour-coded by typology: military sites (red), civilian sites (green) and unspecified (grey)}
\label{britmap}
\end{sidewaysfigure}

The lower similarity of stamp codes in \textit{Britannia} compared to \textit{Baetica} could be explained by the spatial correlation as the area under study is much bigger and the sites are more distant between them. This can be observed more clearly on military sites as the ones with higher similarity are geographically closer. Two patterns can be observed on the dendrogram: 1) nearby sites tend to share more similar stamps and 2) a majority of stamp codes are only found in one site.
 
The similarity of sites based on Dressel 20 stamp codes is also low in \textit{Germania} as can be seen in Figure~\ref{germap}. 

\begin{sidewaysfigure}[htp]
	\centering
\includegraphics[angle=180, width=\linewidth]{figs/dendroger5.pdf}
\caption{Dendrogram displaying the results of hierarchical clustering based on Morisita-Horn metrics for sites in \textit{Germania}. Site names are colour-coded by typology: military sites (red), civilian sites (green) and unspecified (grey)}
\label{germap}
\end{sidewaysfigure}

The results for \textit{Germania} follow a similar pattern than \textit{Britannia} but with a minor correlation as we can observe in the dendrogram.

A higher concentration of sites sharing similar stamps was found in areas eminently militarised and close to German \textit{limes}, although most sites mostly showed different stamps. 

\section{Discussion}

This work has explored to what extent the distribution of amphoric stamps can provide insights into the organisation of olive oil Roman markets both at the production and consumption areas. For this reason, an index of dissimilarity was used to detect differences between the distribution of the amphoric stamps and the spatial distance of producers and consumption centres. 


\subsection{Production centres}

The analysis of the amphora workshops in \textit{Baetica} province suggests that a majority of stamp codes were only used on a single workshop. Beyond this clear pattern, it seems that the similarity of stamp codes is correlated with spatial distance and the amphoric production of spatially close sites shared some stamp codes. Particularly relevant here is the fact that our results show how similar stamp codes were found on sites belonging to the same \textit{conventus} area. This result could indicate that the workshops of the different \textit{conventus} used different stamp codes that were not shared between them. Despite these stamps tend to share the same area of production, we do not identify groups of any more than three workshops sharing the same amphoric stamps as the dendrogram showed. 

Our results do not support previous working hypotheses suggesting that groups of workshops used specific groups of stamp codes; while there is higher stamp similarity for nearby workshops this result does not suggest any kind of organisation beyond the simple fact that nearby workshops had a higher degree of interaction. In fact; the opposite seems true: each workshop used different stamp codes and only shared some of them with the closest ones. 

This unique link between workshop and stamp code can be interpreted in different ways. First, it is possible that each workshop operated independently and did not normally collaborate with other workshops.  Second, stamp similarity in closer workshops could be linked to a spatial pattern. It is more probable than closer workshops tend to share more traits than distant workshops. While the role of the river was significant for the distribution of amphorae, river connectivity amongst workshops does not seem to show relevancy for the distribution or concentration of stamps in a specific river area.

Alternatively, the presence of different stamps in the same workshop could imply some kind of organisation from within the workshop only affecting nearby centres \citep{juanmorostesis}. The stamped amphorae could have allowed potters to organise the different production batches for posterior commercialisation.

This could be explained such as batch systematic organisation between potters \citep{juanmorostesis}. This work method could have allowed potters to identify batches by stamping amphorae for posterior commercialisation. High production of amphorae had to maintain a specialise system organisation to control this production \citep[104]{juanmorostesis}.

%This work method could have allowed to potters to stamp amphorae in order to identify the production for batches for posterior commercialization. 


%This method allowed to potters organises the production with the batch stamping for the posterior commercialisation

It is worth mentioning that the performed analysis could be limited by a diversity of biases. Some workshops have been identified with different names despite the fact that they probably are sections of the same workshop; on the other hand, the amount of stamps collected on different sites is affected by different intensities for field surveys a lack of excavation. As a consequence, further fieldwork could provide a clearer picture of the links between workshops if the sample size of stamp codes is increased.

\subsection{Consumption centres}

Both consumption areas showed a correlation between the spatial distance of sites and their similarity of amphoric stamps. In the case of  \textit{Britannia}, the correlation was higher than \textit{Germania}. In \textit{Britannia} the majority of similarity stamps were mainly found in military centres.  

A combined exploration of the map and the dendrogram also suggests an interesting pattern: centres with higher similarity values are closer to the coast (North Sea and Celtic Sea). This may indicate that the Atlantic route could have played an essential role in transporting olive oil to the area of \textit{Britannia}, since in the places where there is greater similarity they are found in different strategic points near the sea.

This trend is aligned with recent works that have suggested that the Atlantic trade routes were mostly used for military supply~\citep{remesal_annona_1986,remesal_provincial_2008,carreras_atlantic_2012,morillo_hispania_2016,rubio-campillo_provincias_2018}. Therefore, a majority of centres with higher similarity are related to military activities and near the coasts, thus suggesting that Dressel 20 containing olive oil were shipped to military areas and then redistributed to the civilian population using land-based transportation systems~\citep{carreras_britannia_1998,ayllon_olive_2018}.

%The results for \textit{Germania} follow a similar pattern than \textit{Britannia} but with a minor correlation as we can observe in the dendrogram\xavi{esta frase está repetida de los resultados. quizá quitarla de ahí y dejarla aquí?}. 

The areas mostly militarised share similar stamps than civil areas. However, we do not detect a concrete pattern regarding the distribution of the stamps in the German limes \citep{xanten2018}. This could be also interpreted by an intensity bias where military centres have been mostly excavated than civil centres.

%\xavi{pero no decías que el aceite llegaba por las legione? Entonces por qué es un bias? En todo caso ponlo al final de la sección y no al inicio}.

%It is worth mentioning that it was not detected a robust model of organisation in both cases with regard to the distribution of stamps in consumption centres\xavi{de nuevo esto es redundante no?}. 

Finally, the results suggest that there is no clear organisational pattern linking production centres and/or consumption centres beyond the spatial distance. As a consequence, our analysis does not indicate the presence of a specific workshop from \textit{Baetica} supplying to some specific province. 

This evidence means that it is unknown whether some production centres went to one province or another or, at least, that they can be clearly reflected in the data with a greater similarity in the amphoric stamps. Thus, production centres could have distributed randomly olive oil to both \textit{Britannia} and \textit{Germania}. Neither we do not detect production centres dedicated to the distribution of olive oil in a certain province. 

%\xavi{me petaría esta frase, porque entre otras cosas arriba dices que hay correlación espacial, así que cómo es que aquí dices que no hay distribución geográfica aparente?}Judging by the results obtained, there does not seem to be a specific pattern in terms of geographical distribution. Nor it was detected consumption areas where stamps are specified from an amphora workshop. 


\section{Concluding remarks}

This work has used quantitative analysis to explore long-range trade of olive oil using the stamps found in Dressel 20 vessels. The most interesting result is linked to the meaning of these inscriptions. The use of specific codes seems to be almost exclusively decided by each workshop, so it could be used to identify amphorae found in consumption centres (i.e. an amphora with a given stamp was probably made on a specific workshop). This idea could explain why the similarity of stamps between workshops is generally very low. On the other hand, some code stamps have been identified in nearby workshops so this pattern suggests that the clustered workshops were somehow connected (e.g. they belonged to the same owner).

%Alternatively, the presence of different stamps in the same workshop could imply some kind of organisation from within the workshop only affecting nearby centres \citep{juanmorostesis}. The stamped amphorae could have allowed potters to organise the different production batches for posterior commercialisation\xavi{estos 2 párrafos igual deberían ir a la discusión no? La conclusión debería ser algo más general y aquí estás aún interpretando resultados}.  


A large number of Dressel 20 were not marked with any stamps, so some researchers suggest that potters marked the amphorae to prepare and distribute the product in order to be shipped \citep{berni_millet_epigrafianforica_2008}. This method would be used as an identifier to count the number of amphorae of a batch \citep{juanmorostesis}. 

Unique stamp codes could have also served to identify different groups of potters working in the same amphora workshop. Potters could have marked the amphorae to distinguish different groups working at the same time in different orders~\citep{li_crossbows_2014}. This hypothesis could explain why we detect different stamps in the same workshop. In any case, at present we do not have enough archaeological evidence to validate which of these interpretations is more plausible in our case study.

To conclude, the method presented here provides a useful framework to improve our understanding of the organisation and processes that allowed the Roman Empire to manage a massive and highly specialised production of essential goods. This work has identified similarities between amphoric stamps and spatial distribution in the case of the amphoric production within the Roman Empire.

The growing amount of archaeological data requires of large-scale quantitative methods such as the one presented here to identify and explore complex patterns, thus allowing us to infer dynamics and improve our interpretation of complex economic processes of past societies. 

\section{Acknowledgements}

This research was partly funded by the European Research Council Advanced Grant EPNet (340828). XRC is funded by the Ramon y Cajal programme RYC2018-024050-I (Fondo Social Europeo – Agencia Estatal de Investigación).
We are grateful to Simon Carrignon, Juan Moros, Ignacio Morer and Víctor M. Lozano for their useful suggestions.  
Data have been analysed and conducted in R program version 3.2.4, using the packages \textit{vegan} \citep{oksanen_vegan_2007}, \textit{ggplot2} \citep{ggplot2:_2016}. Maps were performed using QGIS software v.3.14 `Pi'. Datasets and source code are freely available under Open licenses from https://github.com/Mcotsar/Baetica\_stamps.


\section{References}

%\bibliography{bibliotex}
\bibliography{bibtesis}


\end{document}
