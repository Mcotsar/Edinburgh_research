\documentclass[review]{elsarticle}
\usepackage{hyperref,lineno}
\usepackage{xcolor}
\usepackage[utf8]{inputenc}
\modulolinenumbers[5]

\newcommand{\memo}[2]{\textcolor{#1}{#2}}
\newcommand{\maria}[1]{\memo{red}{#1\\}}
\newcommand{\revise}[1]{\memo{blue}{#1\\}}

\journal{Journal of Archaeological Science}


\bibliographystyle{model2-names.bst}\biboptions{authoryear}


\begin{document}


\section{Introduction}


All the amphora stamps belong to Dressel 20 types. Dressel 20 was used to transport olive oil along the province during the Roman Empire. There is evidence that stamps were used from three centuries. (economía oleica betica). Frequently, stamps were marked mostly in handles but rarely in rims and body.  
The meaning of stamp is still under debate. Some authors suggest that they were identified as the owner of the olive land. Other authors propose that stamp could be the owner of the amphorae workshop.  



The information of the stamps is known as tria nomina (HABLAR SOBRE EL DEBATE DE LOS INDIVIDUOS DE ACEITE O YO QUE SE)and contend 



\section{Material and Methods}

We analysed a dataset of 131 stamps from 81 workshops. All the dataset was collected from CEIPAC database (citar). The workshops were located in different locations in Baetica province, along the river Guadalquivir and its tributary Genil. 








%los sellos fueron marcados a partir del siglo II d.C (la economía oleícola bética Remesal)

\subsection{Jaccard distance}
%%usar Jaccard disimilarity and similarity

The dataset was analysed using a statistic method as Jaccard index. Jaccard distance was used to calculate the distance among different stamps.  






\end{document}