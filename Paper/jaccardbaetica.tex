\documentclass[review]{elsarticle}
\usepackage{hyperref,lineno}
\usepackage{xcolor}
\usepackage[utf8]{inputenc}
\modulolinenumbers[5]

\newcommand{\memo}[2]{\textcolor{#1}{#2}}
\newcommand{\maria}[1]{\memo{red}{#1\\}}
\newcommand{\revise}[1]{\memo{blue}{#1\\}}

%\journal{Journal of Archaeological Science}


\bibliographystyle{model2-names.bst}\biboptions{authoryear}


\begin{document}

\begin{frontmatter}


\begin{abstract}

The aim of this study is to detect the patterns of olive oil production that link amphora workshops and amphoric stamps. Roman provinces such as Baetica became important production and distribution centers during the Roman Empire. However, it remains under debate how this province was organized and whether it is possible to identify patterns in the olive oil market. 

Our case of study has been focused on the production processes located in Baetica province (currently Andalusia) from 1st to 3rd AD. In particular, we want to explore economic dynamics that include the production and distribution of olive oil trade. Amphoric stamps are used to identify the presence of different producer groups that might share similar stamps. To achieve this goal, we analyse a set of stamps from different workshops in Baetica province in order to detect a relation between the distribution of amphoric stamps and the economic structure in this province. Here we use methods borrowed from Ecology that allow us to identify if amphora workshops share similar amphoric stamps depending on the spatial distance. 

The analysis explores how quantitative approach provides a useful tool for the interpretation of the economic processes. Finally, results pretend to highlight the organization of Baetican olive oil production in the Roman Empire linked to the differences observed in the archaeological evidence.

\end{abstract}


\end{frontmatter}

\section{Introduction}


Material culture allows us to understand a part of mechanism of production... (aquí creo que me repito como siempre)


The economy of Roman Empire have been an object of study in the last centuries. (cambiar)

This paper aims to highlight the production dynamics in relation to a specific area within Roman Empire. We want to detect the pattern of olive oil production that link amphora workshops and amphoric stamps.
We focus here on exploring the economic relation between stamps and amphora production centres. 

Roman provinces such as Baetica became important production and distribution centers during the Roman Empire. However, it remains under debate how this province was organized and whether it is possible to identify patterns in the olive oil market. 

%this paper aims to highlight the production dynamic in connection with the proximity between amphora workshops and stamps. 

All the amphora stamps belong to Dressel 20 types \citep{dressel,martin-kilcher_romischen_1994}. Dressel 20 is commonly associated with transportation of Baetican olive oil through the provinces during the Roman Empire \citep{berni_millet_epigrafianforica_2008}. Most Dressel 20 were marked in stamps and \textit{tituli picti} and inked in graffities with different information but there is not a general consensus about the meaning of them \citep{rodriguez_baetican_1998}. Stamps are the most studied in this type of amphorae. There is evidence that stamps were used for almost three centuries. (economía oleica betica). Frequently, stamps were marked mainly in handles but rarely in rims and body. 

The information of the stamps is shown in different forms and letter content and it seems that there was not a unique criteria (CITAR). Stamps was mostly formed for a code of three letters. There letters can appear in a abbreviated form or complete and they are known as \textit{Tria Nomina} \citep{berni_millet_amphora_1996}. 


The meaning of the amphora stamps is still under debate. Some authors suggest that they were identified as the land-owners of the olive groves \citep{rodriguez_economioleicola_1977}. Other authors propose that stamp could be the owners of the making-amphorae workshop (CITAR) or even a production counting system \citep{berni_millet_epigrafianforica_2008}. In any case, the use of these stamps defined somehow the system of working in the workshops. 
%habría que citar a Dressel
%olive producing

Here, therefore, this study aim to explore the effect of Baetican olive oil production in the Roman Empire
of production organization over Baetican olive oil production. 


\section{Material and Methods}

\subsection{Case study}

Our case study examines the relation between the distribution of amphoric stamps and the workshops. We studied a dataset of 3787 stamps collected from different amphora workshops in Baetica provice (see CEIPAC database). The workshops were situated in different locations in Baetica province, along the river Guadalquivir and its tributary Genil in order to detect similarities between stamps from workshops and spatial distance. 

However, the 70 \% of stamps cannot be tested due to fragmentation or incomplete information. Consequently, we discard integrate the fragmented stamps in our dataset. We finally filter a total sample of 987 stamps composed by 131 different stamps from 81 workshops. 

The chronology in the workshops is widely diverse from the first to the third centuries AD . However, some stamps show a more specific chronology while the majority of them display a large activity of production being difficult to specify an accurate chronology. A reason may be that most of them were partially excavated and only focused on archaeological surveys to collect the maximum stamps as possible (CITAR) 

%los sellos fueron marcados a partir del siglo II d.C (la economía oleícola bética Remesal) hablar de stams más concreto

%%\subsection{Jaccard distance}
%The dataset was analysed using a statistic method as Jaccard distance. This method allows to measure the dissimilarity by calculating the presence of sets (CITAR). In our case, Jaccard distance was used to compute the mutual presence of traits in the amphora stamps but it does not consider the number of absences. A comparison was done with the distance of the workshops to identify whether there was an association between stamps and spatial distance amongst workshops. 

\subsection{Quantifying the diversity Dissimilarity correlation}

%quizás habría que hablar también del filtrado de códigos que se hizo en python porque antes había 3783 stamps pero se filtró por el número de letras...

The approach proposed here is based on the idea of measuring the similarity between amphora workshops by quantifying similar stamps. A measure of dissimilarity has been chosen to analyse the dataset. We use the statistical technique Morisita-Horn index. The formula can be described as follows

(FORMULA Y DESCRIPCIÓN)


This method was performed to measure the overlap between different samples of sets \citep{morisita_measuring_1959, horn_measurement_1966}. In Ecology, it describes the dissimilarity between the system of two communities. 

Considering our dataset as non-uniform sample, this method provides a useful tool to handle large samples with different sizes and diversity \citep{wolda_similarity_1981}. Morisita-Horn index can be expressed considering 0 as total presence of similarity of stamps and 1 a totally dissimilarity between stamps. In our case, it will be calculated the number of times that one stamp appear in a amphora workshop. This method allows to bear in mind the similar number of times for each repeated stamp per workshop. 
%quizas ampliar con el método de xavielin

%hablar sobre Horn (Morisita)hablar sobre que el sample no estaba uniforme por eso usamos el morisita horn

\section{Results}

The analysis shows that amphoric stamps are not correlated with spatial distance. 



A matrix distance of similarity between workshops can be seen in Fig.1. The matrix displays

Morisita-Horn dissimilarities

A dendrogram was generated 

Some closest workshops show a similarity on the stamps such as BLA BLA BLA . This could be due to several reasons. 

doesn't support


As summary, these results

%%\subsection{Jaccard distance}

%Result of Jaccard distance can be The coefficients range from x to x.The dendrogram shown in Fig. Bla bla was obtained with the Jaccard distance measure. The dendrogram suggests 


\section{Discussion and Conclusion}

No strong similarity between stamps and amphora workshops were found. 

There is not connection between stamps and amphora workshops

There is not similar stamps in closest workshop 

Similar stamps were not found in closest amphorae workshops so it could be interpreted as no direct connection between stamps and the location of amphora workshops were found. 

It could be due to several reason. On the one hand, the use of this amphoric stamps were exclusively running by the owner or family to distinguish the amphora workshop (CITAR). On the other, it could be somehow a batch systematic organization to prepare and distribute the commodity, considering that Dressel 20 was not marked in several cases. 


just 30-40 \% of amphora Dressel 20 was marked 
An any case, 
Considering, 

\section{Acknowledgements}

The research was funded by European Research Council Advanced Grant EPNet (340828). We are grateful to 
All data has been analysed and conducted in R program version 3.2.4, using vegan package (citar) 


\section{References}

\bibliography{bibliotex}



\end{document}