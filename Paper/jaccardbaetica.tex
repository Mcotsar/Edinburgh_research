\documentclass[review]{elsarticle}
\usepackage{hyperref,lineno}
\usepackage{xcolor}
\usepackage[utf8]{inputenc}
\modulolinenumbers[5]

\newcommand{\memo}[2]{\textcolor{#1}{#2}}
\newcommand{\maria}[1]{\memo{red}{#1\\}}
\newcommand{\revise}[1]{\memo{blue}{#1\\}}

%\journal{Journal of Archaeological Science}


%\bibliographystyle{model2-names.bst}\biboptions{authoryear}


\begin{document}


\section{Introduction}


All the amphora stamps belong to Dressel 20 types. Dressel 20 was used to transport olive oil along the province during the Roman Empire. There is evidence that stamps were used from three centuries. (economía oleica betica). Frequently, stamps were marked mostly in handles but rarely in rims and body.  
The information of the stamps is known as \textit{tria nomina} (HABLAR SOBRE EL DEBATE DE LOS INDIVIDUOS DE ACEITE O YO QUE SE). \textit{Tria Nomina} was mostly formed for a code of three letters. 

The meaning of the amphora stamps is still under debate. Some authors suggest that they were identified as the owner of the olive land \citep{rodriguez_economioleicola_1977}. Other authors propose that stamp could be the owner of the amphorae workshop. In any case, the use of these stamps defined the system of working in the workshops. 
%habría que citar a Dressel

\section{Material and Methods}

We analysed a dataset of 131 stamps from 81 workshops. All the dataset was collected from CEIPAC database (citar). The workshops were located in different locations in Baetica province, along the river Guadalquivir and its tributary Genil in order to detect similarities between stamps from workshops and spatial distance. The chronology detected in the workshop is widely diverse from I-III B.C. However, all the workshops presented a long activity of production being difficult to specify a accurate chronology. 


%hablar de la cronologia

%los sellos fueron marcados a partir del siglo II d.C (la economía oleícola bética Remesal)


\subsection{Jaccard distance}
%%usar Jaccard disimilarity and similarity

The dataset was analysed using a statistic method as Jaccard distance. This method allows to measure the dissimilarity between sets. In our case, Jaccard distance was used to calculate the mutual presence of traits in the stamps. A comparison was done with the distance of the workshops to know whether there was an association between stamps and spatial distance. 


\subsection{Dissimilarity correlation}


This method was performed to measure the overlap between different samples of sets \citep{horn_measurement_1966}. Unlike Jaccard distance, Morisita-Horn variant can be used to handle large samples with different sizes and diversity. The index can be described  where 0 is no presence of overlap and 1 totally similarity. 
%hablar sobre Horn (Morisita)

\bibliography{bibliotex}


\end{document}