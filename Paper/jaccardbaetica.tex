\documentclass[review]{elsarticle}
\usepackage{hyperref,lineno}
\usepackage{xcolor}
\usepackage[utf8]{inputenc}
\modulolinenumbers[5]

\newcommand{\memo}[2]{\textcolor{#1}{#2}}
\newcommand{\maria}[1]{\memo{red}{#1\\}}
\newcommand{\revise}[1]{\memo{blue}{#1\\}}

%\journal{Journal of Archaeological Science}


%\bibliographystyle{model2-names.bst}\biboptions{authoryear}


\begin{document}


\section{Introduction}


All the amphora stamps belong to Dressel 20 types. Dressel 20 was used to transport olive oil along the province during the Roman Empire. There is evidence that stamps were used from three centuries. (economía oleica betica). Frequently, stamps were marked mostly in handles but rarely in rims and body.  
The information of the stamps is known as \textit{tria nomina} (HABLAR SOBRE EL DEBATE DE LOS INDIVIDUOS DE ACEITE O YO QUE SE). \textit{Tria Nomina} was mostly formed for a code of three letters. 

The meaning of the amphora stamps is still under debate. Some authors suggest that they were identified as the owner of the olive land \citep{rodriguez_economioleicola_1977}. Other authors propose that stamp could be the owner of the amphorae workshop. In any case, the use of these stamps defined the system of working in the workshops. 
%habría que citar a Dressel

\section{Material and Methods}

We analysed a dataset of 990 stamps from 81 workshops. From the dataset, a sample of 131 different stamps were identified and collected from from CEIPAC database (citar). The workshops were situated in different locations in Baetica province, along the river Guadalquivir and its tributary Genil in order to detect similarities between stamps from workshops and spatial distance. The chronology in the workshops is widely diverse from I-III B.C. However, some stamps show a more specific chronology while the majority of them display a large activity of production being difficult to specify an accurate chronology. A reason can be that most of them were partially excavated focused on archaeological surveys to collect the maximum stamps as possible (CITAR) 

%los sellos fueron marcados a partir del siglo II d.C (la economía oleícola bética Remesal) hablar de stams más concreto


\subsection{Jaccard distance}
%%usar Jaccard disimilarity and similarity

The dataset was analysed using a statistic method as Jaccard distance. This method allows to measure the dissimilarity by calculating the presence of sets (CITAR). In our case, Jaccard distance was used to compute the mutual presence of traits in the amphora stamps but it does not consider the number of absences. A comparison was done with the distance of the workshops to identify whether there was an association between stamps and spatial distance amongst workshops. 

%indicar si se usa en archaeology el jaccard distance

\subsection{Dissimilarity correlation}

As a complement of Jaccard distance, we use the statistical technique Morisita-Horn index. Unlike Jaccard distance, this method was performed to measure the overlap between different samples of sets \citep{horn_measurement_1966}. In ecology, it describes the dissimilarity between the system of two communities. 

Considering our dataset as non-uniform sample, this method provides a useful tool to handle large samples with different sizes and diversity. Index can be describe considering 0 as no presence of similarity of stamps and 1 a totally similarity between stamps. 


%hablar sobre Horn (Morisita)hablar sobre que el sample no estaba uniforme por eso usamos el morisita horn

\section{Results}

\section{Discussion and Conclusion}

No similarity between stamps and amphora workshops were founded. 

\section{Acknowledgements}

The research was funded by European Research Council Advanced Grant
EPNet (340828). 


\bibliography{bibliotex}


\end{document}